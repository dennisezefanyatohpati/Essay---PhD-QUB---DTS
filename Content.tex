\vspace{4mm}

Computational technologies have been improving exponentially for the past decades and have become an integrated part of engineering. I have always had a fascination with aerospace engineering and computer science, and I believe the combination of both fields could create a limitless impact on engineering solutions.
I see that the project titled "Integration of Computational Fluid Dynamics, Artificial Intelligence, and Metrology to understand how Uncertainty (Manufacturing and Modelling) Can Affect Nozzles Manufacturing and Performance" possesses three crucial aeronautical engineering facets in the 21st century which are: manufacturing technologies, computer science, and metrology which known as the key of engineering efficiency and development especially in the aeronautics field. 

Technologies development to improve accuracy in manufacturing aircraft components has dramatically improved yet affect its performance. High-fidelity computational simulation development will allow manufacturers to capture the effect of diversity between digital and actual geometry and significantly reduce the overall cost and time. My big vision on its future use of enhancing the manufacturing efficiency as its knowledge, purpose, and opportunities for creative and collaborative across multiple fields that align with my passion and future career plan has become my biggest motivation for this research. Side-by-side work with Aircraft Research Association (ARA) and Dr Marco Geron will also give me an immediate constructive criticism and expert's perspective of my work, allowing me to develop a better research quality in the field.

On my second year, I had a chance to lead the aircraft design team competition. Through this project, I have experienced a theoretical and practical research aspect in a very diverse area such as structure, aerodynamics, and manufacturing to market research. As I enter my final year, I am excited to be challenged for more technical work. Worked on the AIAA strengthen my interest in aerodynamics world. I chose to work on an aerodynamic team with Queen's Formula Student to design and  manufacture the racecar's undertray as my final year project. With profound understanding in aerodynamics, I found it riveting to go through different fluid physics modelling and how it could affect the flow behaviour.  My fascination in the art of CFD and its complexity pushed me to go beyond my project by exploring more variables to understand and improve the aerodynamics undertray's performance. This project has given me invaluable insight into theoretical and practical CFD and a fond opportunity to sharpen my skills and experience in engineering practice for my this doctoral project.

Outside school, I have developed an interest in fundamental of computer programming in purpose to accelerate my projects and study. Through practical application along with my studies and work, I have developed a fluency in some programming language such as MATLAB, Python, and C++, which I believe will significantly support the project in the computational aspect.

In summary, by the end of this research, I would have gained the comprehension, technical knowledge, and scientific knowledge I need to unite the theory and practice in my later career, and this will be a leap step towards my bigger goals as an engineers and a leader.



